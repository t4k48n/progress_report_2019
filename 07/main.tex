% 報告会テンプレート(LuaLaTeX対応版)
\documentclass[10pt,a4j,twocolumn]{ltjsarticle}


\usepackage{graphicx}   % \includegraphics を使うためのパッケージ読み込み
\usepackage{imcreport}  % IMC報告会用テンプレートパッケージの読み込み

\usepackage{amsmath,amssymb}
\usepackage{bm}

% ハイフネーションを抑制する.数値が大きいほど,無理にでも抑制しようとする.
\hyphenpenalty=10000\relax
\exhyphenpenalty=10000\relax
\sloppy


\title{不確かさの影響を評価関数に取り入れた冗長マニピュレータの軌道生成} % タイトル
\author{高谷 秀明}                              % 著者
\studentnumber{TD18K001}                         % 学籍番号
\date{2019-07-23}                              % 日付
\header{前期最終報告}                                % 左上のヘッダ内容


\begin{document}

\maketitle % タイトル,著者情報などを挿入

\section{はじめに}

前回は不確かさを持つ動的マニピュレータについて,不確かさが動的可操作性に及ぼす影響を調査した.
その結果,動的可操作性は不確かさを用いた動的マニピュレータの姿勢決定には利用できないと結論づけた.

\section{不確かさを持つ動的マニピュレータ}

\begin{equation}
  \bm{M}
\end{equation}

\section{軌道最適化問題}

与えられた時間$\varDelta T$で初期姿勢$\bm{q}_{i}$および終端姿勢$\bm{q}_{f}$へ姿勢変更するための姿勢軌道$\bm{q}(t) \; (0 \leq t \leq \varDelta T)$を求める.

\end{document}
